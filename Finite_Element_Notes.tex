\documentclass[10pt]{report}
\usepackage{stmaryrd}
\usepackage{amsmath}
\usepackage{amsfonts}
\usepackage{amssymb}
\usepackage{fullpage}
\usepackage{graphicx}
\numberwithin{equation}{section}


\title{Finite Element Method Notes}

\author{Michael Harmon}


\begin{document}

\maketitle

\tableofcontents

\chapter{Finite Element Methods}


\noindent
The finite element method is a method for approximating solutions to
boundary value problems that has gained much popularity in the realm of 
numerical methods for partial differential equations.  While finite element methods can be applied to a range of boundary value problems we focus our attention on Poisson's equation:

\begin{equation}\label{eq:Poisson}
-\nabla \, \cdot \, \left(\epsilon \, \nabla \, \Phi \right) \, = \, f \qquad \text{in} \quad \Omega
\end{equation}


\noindent
Where $ 0 < \epsilon_{*} 
\leq \epsilon(x)  \leq \epsilon^{*} $ is bounded positive-definite function. And the boundary conditions are:


\begin{align}
\Phi \, &= \, \Phi_{D} \qquad \text{on} \quad \partial \Omega_{D} \label{eq:DirichletBC} \\
-\nabla \Phi \, \cdot \, \boldsymbol \eta  &= \Phi_{N} \qquad \text{on} \quad \partial \Omega_{N} \label{eq:NeumannBC} \\
 \partial \Omega = \partial \Omega_{D} \, \cup \, \partial \Omega_{N} \quad & \quad \partial \Omega_{D} \, \cap \, \partial \Omega_{N} = \emptyset
\end{align}


\noindent
Where $\partial \Omega_{D}$ correspond to Dirichlet portions boundary and $\partial \Omega_{N}$ corresponds to Neumann portions of the boundary.  We require that $\Phi_{D} \in H^{1/2}(\partial \Omega_{D})$, $\Phi_{N} \in H^{1/2}(\partial \Omega_{N})$, and $f \in H^{-1}(\Omega)$.
 


\vspace{2mm}


\noindent
Finite element methods seek \textit{weak solutions} of differential equation $\Phi \in H^{1}(\Omega)$ by multiplying our equation by a test function $v \in H^{1}(\Omega)$ and integrating over entire the domain

$$ \int_{\Omega} \, - v \, \nabla  \, \cdot \, \left(\epsilon \, \nabla \, \Phi \right) \, dx \, =  \, \int_{\Omega} \, v \, f \, dx $$

\noindent
Using integration by parts on the left hand side this becomes


\vspace{2mm}


\begin{equation}
\int_{\Omega} \, \nabla  \, v  \, \cdot \, \left(\epsilon \, \nabla \, \Phi \right) \, \, dx \, 
-\int_{\partial \Omega_{D}} \, v \, \, \left(\epsilon \, \nabla \, \Phi \, \cdot \, \boldsymbol \eta \right) \, dx \,  
-\int_{\partial \Omega_{N}} \, v \, \, \left(\epsilon \, \nabla \, \Phi \, \cdot \, \boldsymbol \eta \right) \, dx \,  
=  \, \int_{\Omega} \, v \, f \, dx 
\end{equation}


\vspace{2mm}


\noindent
Using the condition of \eqref{eq:NeumannBC} we have  the Neumann boundary condition to be  $\Phi_{N}$:


\vspace{2mm}


$$
\int_{\Omega} \,  \,  \nabla \, v \, \cdot \, \left(\epsilon \, \nabla \, \Phi \right) \,  dx \, 
-\int_{\partial \Omega_{D}} \,  \,  v \,\left(\epsilon \, \nabla \, \Phi \, \cdot \, \boldsymbol \eta \right) \,dx \,  
=  \, \int_{\Omega} \, v \, f \,   dx  \, - \, \int_{\partial \Omega_{N}} \,  v \, \Phi_{N} \, dx
$$


\noindent
Our problem \eqref{eq:Poisson}, \eqref{eq:DirichletBC} and \eqref{eq:NeumannBC}  then becomes, 

\begin{center}
Find $\Phi-\Phi_{D} \in V = \{ \phi \in H^{1}(\Omega): \phi = 0 \quad \text{on} \quad \partial \Omega_{D} \}$. Such that for all $v \, \in \, V$ we have:
\end{center}


\begin{equation} \label{eq:PoissonWeakForm}
\int_{\Omega} \, \epsilon(x) \, \left(  \nabla \, \Phi \cdot \,
 \nabla \, v \, \right) \, dx \, 
=  \, \int_{\Omega} \,  f \, v \, dx  \, - \, \int_{\partial \Omega_{N}} \,  v \, \Phi_{N} \, dx
\end{equation}

\vspace{3mm}


\noindent
Note, that we have turned our differential equation in \eqref{eq:Poisson} into a variational problem in \eqref{eq:PoissonWeakForm}.  We note our solution $\Phi$ and our test function $v$ both come from the same function space $H^{1}(\Omega)$, when this occurs our formulation is known as a \textit{Galerkin formulation}.   


In the above we have shown that \textit{strong solutions}, that is solutions to, \eqref{eq:Poisson}, \eqref{eq:DirichletBC}, and \eqref{eq:NeumannBC} satisfy the variational formulation \eqref{eq:PoissonWeakForm}.  Going the other direction, one can show that with appropriate regularity conditions on $\Phi_{D}, \Phi_{N}$ and $f$,  solutions to \eqref{eq:PoissonWeakForm} satisfy \eqref{eq:Poisson}, \eqref{eq:DirichletBC}, and \eqref{eq:NeumannBC}.  This is important, because in it will show that our finite element methods (which are based of the variational form of Poisson's equation) are \textit{consistent}.  Consistency of a numerical method is a necessity to show that the method converges, i.e. our numerical approximations to the solution converge to the true solution.  


\vspace{5mm}

\noindent
Calling our function space $V = H^{1}(\Omega)$, we can rewrite the above using a bilinear the functional $B: \, V \times V \rightarrow \mathbb{R}$ and our problem becomes

\begin{center}
Find $\Phi - \Phi_{D} \in V $ such that,
\end{center}

\begin{equation}\label{eq:PoissonBilinear}
B \left( v \, , \Phi \, \right) \,  
=  F(v) \qquad \forall \,  v  \, \in  V 
\end{equation}


\noindent
Where 

$$B\left( \, v \, , \Phi \, \right) \, =
  \int_{\Omega} \, \epsilon(x) \, \left( \nabla \, v \, ,  \nabla  \, \Phi  \,
 \right) \, dx \, $$ 
 
 
\noindent
And 


$$ F(v)  =  \, \int_{\Omega} \, v \,  f  \, dx   \, - \, \int_{\partial \Omega_{N}} \,  v  \, \Phi_{N} \, dx$$


In order to prove existence and uniqueness of a \textit{solution} $\Phi  \in V$ as well those to $\Phi \in V + \Phi_{D}$ we must show that our Galerkin formulation satisfies the requirements of the \textit{Lax-Milgram theorem}.  First we show $B(\cdot, \cdot) $ is \textit{continuous (bounded)} by Cauchy-Schwartz inequality:



\begin{align*}
|B(v , \Phi)| &= \left\vert \, \int_{\Omega} \epsilon(x) \nabla v \cdot \nabla \Phi \, dx \right\vert  \\
&\leq \Vert \epsilon \Vert_{L_{\infty}} \, \Vert \nabla v \Vert|^{2}_{L_{2}(\Omega)} \, \Vert \nabla \Phi \Vert^{2}_{L_{2}} \,  \\
&\leq  \epsilon^{*} \,  \Vert v \Vert_{H^{1}(\Omega)} \, \Vert \Phi \Vert_{H^{1}(\Omega)} 
\end{align*}


\noindent
And that $B$ is \textit{coercive} by Poincare's inequality:

\begin{align*}
|B(\Phi, \Phi)| &= \left\vert \, \int_{\Omega} \epsilon(x) \, \left( \nabla \Phi \right)^{2}, dx \right\vert  \\
&\geq \,\text{inf}_{\Omega} | \epsilon | \, \Vert \nabla \Phi \Vert^{2}_{L_{2}(\Omega)} \\
&\geq \,  \epsilon_{*} \, C_{p}  \, \Vert \Phi \Vert_{H^{1}(\Omega)}^{2}
\end{align*}

\noindent
Where $C_{p}$ is the Poincare constant and depends on $\Omega$.  Additionally, we must show that our right hand side functional, $F(v)$, is a continuous function of $v \in V$,

\begin{align*}
|F(v)| &= \left\vert  \, \int_{\Omega} \, v \,  f  \, dx   \, - \, \int_{\partial \Omega_{N}} \,  v  \, \Phi_{N} \, dx \, \right\vert \\
&\leq \, \Vert v \Vert_{L^{2}(\Omega)}^{2} \, \Vert f \Vert_{L^{2}(\Omega)}^{2} + \Vert v \Vert_{L^{2}(\partial \Omega_{N})}^{2} \, \Vert \Phi_{N} \Vert_{L^{2}(\partial \Omega_{N})}^{2} \\
&\leq \, \Vert v \Vert_{L^{2}(\Omega)}^{2} \, \Vert f \Vert_{L^{2}(\Omega)}^{2} + \Vert v \Vert_{H^{1}(\Omega)}^{2} \, \Vert \Phi_{N} \Vert_{L^{2}(\partial \Omega_{N})}^{2} 
\end{align*}

\noindent
Where we have used the Trace Theorem on the last line.  Using $\Vert v \Vert_{L^{2}(\Omega)}^{2} \, \leq \, \Vert v \Vert_{H^{1}(\Omega)}^{2}$ we then have,


\begin{align*}
|F(v)| \, & \leq  \, \left(  \Vert f \Vert_{L^{2}(\Omega)}^{2} \, + \, 
\Vert \Phi_{N} \Vert_{L^{2}(\partial \Omega_{N})}^{2} \right) \, \Vert v \Vert_{H^{1}(\Omega)}^{2} \\
&\leq  \, C_{F}  \, \Vert v \Vert_{H^{1}(\Omega)}^{2}
\end{align*}


Now that we have shown solutions to \eqref{eq:PoissonWeakForm} exist and are unique we can discuss how to approximate those solutions with Finite Element methods.   In order to approximate solutions of $\Phi$ in we must project our infinite dimensional test space $V \, = \, H^{1}(\Omega)$ on to a finite dimensional one $V_{h} = \text{span}\{\psi_{1}, \psi_{2}, \ldots , \psi_{N}\}$, where $N$ is the dimension of our vector space, our problem \eqref{eq:PoissonWeakForm} then transforms into the finite dimensional problem:

\begin{center}
Find $\Phi_{h}-\Phi_{D} \in V_{h}$ such that $\forall \,  v_{h} \, \in  V_{h}$:
\end{center}

\begin{equation} \label{eq:FEMPoisson}
\int_{\Omega} \, \epsilon(x) \, \left(  \nabla \, v_{h} \, \cdot \,  \nabla  \, \Phi_{h} \, 
\right) \, dx \, 
=  \, \int_{\Omega}  \, v_{h} \,  f \, dx \, - \, \int_{\partial \Omega_{N}} \,  v_{h}  \, \Phi_{N} \, dx
\end{equation}

\vspace{3mm}


\noindent
We can rewrite the above using a  bilinear the functional $B: \, V_{h} \times V_{h} \rightarrow \mathbb{R}$, 

\begin{center}
Find $\Phi_{h}-\Phi_{D} \in V_{h}$ such that $\forall \,  v_{h} \, \in  V_{h}$:
\end{center}

\begin{equation}\label{eq:PoissonBilinear}
B \left( v_{h} \, , \Phi_{h} \, \right) \,  
=  F(v_{h})
\end{equation}


\noindent
Where 

$$B\left( \, v_{h} \, , \Phi_{h} \, \right) \, =
  \int_{\Omega} \, \epsilon(x) \, \left( \nabla \, v_{h} \, ,  \nabla  \, \Phi_{h} \,
 \right) \, dx \, $$ 
 
 
\noindent
And 


$$ F(v_{h})  =  \, \int_{\Omega} \, v_{h} \,  f  \, dx   \, - \, \int_{\partial \Omega_{N}} \,  v_{h}  \, \Phi_{N} \, dx$$


\vspace{3mm}

\noindent
By writing defining $\Phi_{D} = \sum \, \Phi_{j} \, \psi_{j}$ and $v = \psi_{i}$ we arrive at linear system of equations:

$$ A \textbf{x} \, = \, \textbf{F} $$


\noindent
Where,

$$ A_{i,j} \, = \,   \int_{\Omega} \, \epsilon(x) \, \left(  \nabla  \, \psi_{i} \, \cdot \, \nabla \, \psi_{j} \right) \, dx \, $$


$$ \textbf{F}_{i} \, =  \, \int_{\Omega} \,  \psi_{i} \,  f \,dx $$


$$ \textbf{x}_{i} \, = \Phi_{i} $$



\vspace{5mm}

\noindent
We remark that the coercivity and continuity of $B$ and continuity of $F$ on $v \in V$ also hold on $v_{h} \in V_{h} \subset V$ ensuring existing and uniqueness of solutions to the finite dimensional problem. The coercivity and symmetry of $B$ ensure that our matrix $A$ is \textit{symmetric positive definite.}


In numerical methods setting, the coercivity of $B$ is also known as \textit{stability}.  Stability of a numerical method is important because the \textit{Lax-Equivalence Theorem} states, 

\begin{center}
If your numerical method is consistent and stable then it is convergent.
\end{center}

\noindent
We have shown that our variational formulation \eqref{eq:PoissonWeakForm} is consistent and therefore our numerical method is consistent. This means the true solution, $\Phi$ satisfies the numerical method,

\begin{equation}
\label{eq:PoissonConsistency}
B\left( \, v_{h} \, , \Phi \, \right) \, = \, F(v_{h}), \quad \forall v_{h} \in V_{h}
\end{equation}

The coerecivity of $B$ shows that our method is stable and therefore says that our numerical method is convergent.  

\vspace{2mm}

We can prove convergence of our numerical method  formally by deriving error estimates for our approximation $\Phi_{h}$ of $\Phi$ by introducing a projection operator $P : V \rightarrow V_{h}$ that satisfies,

\begin{equation}
\Vert \Phi - P \Phi \Vert_{L^{2}}(\Omega) \; \leq \; Ch^{k+1}  \qquad \text{and} \qquad 
\Vert \Phi - P \Phi \Vert_{H^{1}}(\Omega) \; \leq \; Ch^{k} \label{eq:ApproxProp}
\end{equation}

\noindent
Then subtracting \eqref{eq:PoissonBilinear} from \eqref{eq:PoissonConsistency} we have,

\begin{equation}
B\left( \, v_{h} \, , \Phi \,  - \Phi_{h} \right) \, = \, 0, \quad \forall v_{h} \in V_{h}
\end{equation}

\noindent
Then adding and subtracting $P \Phi$ we have,

\begin{equation}
B\left( \, v_{h} \, , \Phi_{h} \,  - P \Phi \right)  \, = \, B\left( \, v_{h} \, , \Phi \,  - P \Phi \right)    \quad \forall v_{h} \in V_{h}
\end{equation}

\noindent
Choosing $v_{h} = \Phi_{h} - P \Phi \, \in V_{h}$ we have,

\begin{equation}
B\left( \, \Phi_{h} \,  - P \Phi  \, , \Phi_{h} \,  - P \Phi \right)  \, = \, B\left( \, \Phi_{h} \,  - P \Phi \, , \Phi \,  - P \Phi \right)   
\end{equation}

\noindent
Using the coercivity and continuity of $B$ results in,

\begin{equation}
\epsilon_{*} C_{P} \Vert \Phi_{h} \,  - P \Phi  \Vert^{2}_{V} \, \leq \, \epsilon^{*} \, \Vert \Phi_{h} \,  - P \Phi  \Vert_{V}  \, \Vert \Phi \,  - P \Phi  \Vert_{V} 
\end{equation}

\noindent
Dividing both sides by $\epsilon_{*} C_{P} \Vert \Phi_{h} \,  - P \Phi  \Vert_{V}$ yields,

\begin{align}
\Vert \Phi_{h} \,  - P \Phi  \Vert_{V} \, &\leq \, \frac{\epsilon^{*}}{\epsilon_{*} C_{P} } \, \Vert \Phi_{h} \,  - P \Phi  \Vert_{V}  \, \Vert \Phi \,  - P \Phi  \Vert_{V}  \\
\Vert \Phi_{h} \,  - P \Phi  \Vert_{V} \,  &\leq \, \frac{\epsilon^{*}}{\epsilon_{*} C_{P} } \, \
h^{k}
\end{align}

Alternatively, we note that $\Vert \Phi_{h} \,  - P \Phi  \Vert_{L^{2}} \, \leq \, \Vert \Phi_{h} \,  - P \Phi  \Vert_{L^{2}} $  and that Cauchy Schwartz,

$$B\left( \, \Phi_{h} \,  - P \Phi \, , \Phi \,  - P \Phi \right)  \,
\leq \, \epsilon^{*} \, \Vert \Phi_{h} \,  - P \Phi  \Vert_{L^2}  \, \Vert \Phi \,  - P \Phi  \Vert_{L^2} $$

\noindent
Which gives us,

\begin{align}
\Vert \Phi_{h} \,  - P \Phi  \Vert_{L^{2}(\Omega)} \, &\leq \, \frac{\epsilon^{*}}{\epsilon_{*} C_{P} } \, \Vert \Phi_{h} \,  - P \Phi  \Vert_{L^{2}(\Omega)}  \, \Vert \Phi \,  - P \Phi  \Vert_{L^{2}(\Omega)}  \\
\Vert \Phi_{h} \,  - P \Phi  \Vert_{L^{2}(\Omega)} \,  &\leq \, \frac{\epsilon^{*}}{\epsilon_{*} C_{P} } \, \
h^{k+1}
\end{align}


\noindent
As $h \rightarrow 0$ we can see that our approximation $\Phi_{h}$ converges to $\Phi$ in $L^{2}(\Omega)$ and $H^{1}(\Omega)$.



\vspace{5mm}



\chapter{Time Dependent Error Estimates}


\begin{align}
\partial_{t} u \ - \ \boldsymbol \nabla \cdot ( D(\textbf{x}) \boldsymbol \nabla u) \; &= \; f(\textbf{x},t) &&  \Omega \times (0,T]  \\
(-D \boldsymbol \nabla u ) \cdot \boldsymbol \eta_{N} \; &= \; g(u) && \partial \Omega_{N} \times (0,T]  \\
u \; &= \; u_{D}(\textbf{x}) &&  \partial \Omega_{D} \times (0,T] \\
u \; &= \; u_{I}(\textbf{x}) &&  \Omega \times \left\{ t=0 \right\}
\end{align}


Where $0 < D_{*} \leq D(\textbf{x}) \leq D^{*}$.  The boundary condition $g: H^{1}(\Omega) \rightarrow L^{2}(\partial \Omega_{N})$ is Lipschitz function such that,
\begin{equation}
\Vert g(u) - g(v) \Vert_{L_{2}(\partial \Omega_{N})} \; \leq \; L
\Vert u - v \Vert_{H^{1}(\Omega)} \qquad \qquad \forall u,v \in H^{1}(\Omega).  \qquad L > 0.
\end{equation}



Let $\mathcal{T}_{h} = \left\{ \Omega_{e} \right\}_{e}$ be the triangulation of $\Omega$. The boundary $\partial \Omega = \partial \Omega_{D} \cup \partial \Omega_{N}$ is smooth enough such that the trace identity,


\begin{equation}
\Vert v \Vert^{2}_{L^{2}(\partial \Omega)} \, \leq \, C_{tr}(\Omega) \Vert v \Vert^{2}_{H^{1}(\Omega)}, \qquad \qquad 
\forall v \in H^{1}(\Omega) \label{eq:Trace}
\end{equation}


\noindent
holds.  The semidiscrete finite element method is, \newline

Find $u_{h}(t) \in \, H^{1}_{D}(\Omega) + u_{D}$ such that,
\begin{align}
(v,\partial_{t} u_{h})_{\Omega} \ + \ B(v,u_{h}) \; &= \; (v,f)_{\Omega} \ - \ \langle v, g(u_{h}) \rangle_{\partial \Omega_{N}} && \forall v \in V_{h} \quad   a.e. \ t \in (0,T], \label{eq:fem} \\
(v,u_{h}(t=0)_{\Omega} \; &= \; (v,u_{I})_{\Omega} && \forall v \in V_{h} \label{eq:fem_init}.
\end{align}


Where $V_{h}$ is the finite dimensional subspace of the Hilbert space $H_{D}^{1}(\Omega) \; = \; \left\{ v \in H^{1}(\Omega) : v \left\vert_{\partial \Omega_{D}} \right.  =  0 \right\}$.  Let be $V_{H}$ be the such that $v \ \in V_{h}$ then $v \vert_{\Omega_{e}} \in Q_{k}(\Omega_{e})$ where $Q_{k}$ is the tensor product of polynomials of order $k$. The bilinear $B : H^{1} \times H^{1} \rightarrow \mathbb{R}$ in \eqref{eq:fem} is coercive, meaning there exists an $\alpha \in \mathbb{R}$ such that,


\begin{equation}
B(v,v) \; \geq \alpha^{2} \; \Vert v \Vert_{H^{1}}^{2} \qquad \qquad \forall v \in H^{1}(\Omega)
\end{equation}

\bigskip


\textbf{Error Estimate:}
\textit{
If $\alpha^{2} \geq 2 \sqrt{L} C_{tr}$ then the finite element method defined in \eqref{eq:fem} and \eqref{eq:fem_init} has errors such that,}
\begin{equation}
\Vert u - u_{h} \Vert^{2}_{L^{\infty}\left([0,T]; L^{2}(\Omega)\right)}  \; \leq \; Ch^{k},
\end{equation}
\textit{for some constant $C = C(T,\alpha^{2},\Omega,L) > 0$.}

\medskip


\underline{Proof}:



Let $P: H^{1}_{D} \rightarrow V_{h}$ be the projection that satisfies the elliptic projection property,

\begin{equation}
B(v,u - Pu) \; = \; 0 \qquad \qquad \forall v \in V_{h}
\end{equation}


And has the approximation properties,

\begin{equation}
\Vert u - Pu \Vert_{L^{2}}(\Omega) \; \leq \; Ch^{k+1}  \qquad \text{and} \qquad 
\Vert u - Pu \Vert_{H^{1}}(\Omega) \; \leq \; Ch^{k} \label{eq:ApproxProp}
\end{equation}



The function $u(t) \in H^{1} + u_{D}$ also satisfies the finite element method, 
\begin{align}
(v,\partial_{t} u)_{\Omega} \ + \ B(v,u) \; &= \; (v,f)_{\Omega} \ - \ \langle v, g(u) \rangle_{\partial \Omega_{N}} && \forall v \in V_{h} \qquad a.e. \ t \ \in (0,T] \label{eq:weak} \\
(v,u(t=0)_{\Omega} \; &= \; (v,u_{I})_{\Omega} && \forall v \in V_{h} \label{eq:weak_init}
\end{align}



Subtracting the \eqref{eq:fem} from \eqref{eq:weak} we have,

\begin{equation}
(v,\partial_{t} (u - u_{h})) + B(v,u-u_{h}) \; \leq \; \vert \langle v, g(u) -g(u_{h}) \rangle_{\partial \Omega_{N}} \vert \qquad \qquad a.e. \ t
\end{equation} 



Define $e_{I}(t) = u(t) - Pu(t)$ and $e_{A}(t) = Pu(t) - u_{h}(t)$ and using the elliptic projection property then the above becomes,

\begin{equation}
(v,\partial_{t} e_{A}) + B(v, e_{A}) \; \leq \; \vert (v,\partial_{t} e_{I}) \vert \ + \ \vert \langle v, g(u) -g(u_{h}) \rangle_{\partial \Omega_{N}} \vert \qquad \qquad a.e. \ t
\end{equation} 


Taking $v = e_{A}(t)$ and using the coercivity property of $B(\cdot, \cdot)$ we have,

\begin{align}
\frac{d}{dt} \Vert e_{A} \Vert^{2}_{L^{2}(\Omega)}  +  2\alpha^{2} \Vert e_{A} \Vert_{H^{1}(\Omega)}^{2}  \leq \;  2 \, \vert (e_{A},\partial_{t} e_{I}) \vert \ + \ 2 \, \vert \langle e_{A}, g(u) -g(u_{h}) \rangle_{\partial \Omega_{N}} \vert \qquad \qquad a.e. \ t
\end{align} 


We now use Young's inequality to obtain,

\begin{align}
\frac{d}{dt} \Vert e_{A} \Vert^{2}_{L^{2}(\Omega)}  +  2\alpha^{2} \Vert e_{A} \Vert_{H^{1}(\Omega)}^{2}  \leq \; \Vert  e_{A} \Vert_{L^{2}(\Omega)}^{2} + 4 \Vert \partial_{t} e_{I} \Vert^{2}_{L^{2}(\Omega)} + \frac{1}{\delta} \Vert  e_{A} \Vert_{L^{2}(\partial \Omega_{N})}^{2} + \delta \Vert g(u) -g(u_{h}) \Vert_{L^{2}(\partial \Omega_{N})}^{2}
\end{align} 



Using the Lipschitz property of $g(\cdot)$ we have,

\begin{align}
\frac{d}{dt} \Vert e_{A} \Vert^{2}_{L^{2}(\Omega)}  +  2\alpha^{2} \Vert e_{A} \Vert_{H^{1}(\Omega)}^{2}  \leq \; \Vert  e_{A} \Vert_{L^{2}(\Omega)}^{2} + 4 \Vert \partial_{t} e_{I} \Vert^{2}_{L^{2}(\Omega)} + \frac{1}{\delta} \Vert  e_{A} \Vert_{L^{2}(\partial \Omega_{N})}^{2} + \delta L \Vert u -u_{h} \Vert_{H^{1}(\Omega)}^{2}
\end{align} 


Using the Trace identity \eqref{eq:Trace} on the second to last term and the triangle inequality on the last term,

\begin{align}
\frac{d}{dt} \Vert e_{A} \Vert^{2}_{L^{2}(\Omega)}  +  2\alpha^{2} \Vert e_{A} \Vert_{H^{1}(\Omega)}^{2}  \leq \; \Vert  e_{A} \Vert_{L^{2}(\Omega)}^{2} + 4 \Vert \partial_{t} e_{I} \Vert^{2}_{L^{2}(\Omega)} + \left(\frac{C_{tr}^{2}}{\delta} + \delta L \right) \Vert  e_{A} \Vert_{H^{1}(\Omega)}^{2} + \delta L \Vert e_{I} \Vert_{H^{1}(\Omega)}^{2}
\end{align} 



Rearranging terms we have,


\begin{align}
\frac{d}{dt} \Vert e_{A} \Vert^{2}_{L^{2}(\Omega)}  +  \left(2\alpha^{2} - \frac{C_{tr}^{2}}{\delta} - \delta L \right) \Vert e_{A} \Vert_{H^{1}(\Omega)}^{2}  \leq \; \Vert  e_{A} \Vert_{L^{2}(\Omega)}^{2} + 4 \Vert \partial_{t} e_{I} \Vert^{2}_{L^{2}(\Omega)} + \delta L \Vert e_{I} \Vert_{H^{1}(\Omega)}^{2} \label{eq:intermediate}
\end{align} 


Taking $ 2\alpha^{2} - \frac{C_{tr}^{2}}{\delta} - \delta \; = \; \alpha^{2}$ we obtain the quadratic form,


\begin{equation}
L \delta^{2} - \alpha^{2} \delta + C_{tr}^{2} \; = \; 0,
\end{equation}


which has a real positive solution if $\alpha^{2} \, \geq \, 2 \sqrt{L} C_{tr}$.  With these choices then \eqref{eq:intermediate} becomes


\begin{equation}
\frac{d}{dt} \Vert e_{A} \Vert^{2}_{L^{2}(\Omega)}  +  \alpha^{2}  \Vert e_{A} \Vert_{H^{1}(\Omega)}^{2}  \leq \; \Vert  e_{A} \Vert_{L^{2}(\Omega)}^{2} + 4 \Vert \partial_{t} e_{I} \Vert^{2}_{L^{2}(\Omega)} + \delta \left( \alpha^{2}, C_{tr} \right) L \Vert e_{I} \Vert_{H^{1}(\Omega)}^{2} 
\end{equation}


Integrating over $(0,t)$ with $t \leq T$ yields,

\begin{align}
\Vert e_{A}(t) \Vert^{2}_{L^{2}(\Omega)}  + \alpha^{2} \int_{0}^{t}   \Vert e_{A} \Vert_{H^{1}(\Omega)}^{2} ds  &\leq \; \int_{0}^{t}  \Vert  e_{A} \Vert_{L^{2}(\Omega)}^{2} ds \\ 
 & \quad + \int_{0}^{t} \left( \Vert e_{A}(0) \Vert^{2}_{L^{2}(\Omega)} + 4 \Vert \partial_{t} e_{I} \Vert^{2}_{L^{2}(\Omega)} + \delta \left( \alpha^{2}, C_{tr} \right) L \Vert e_{I} \Vert_{H^{1}(\Omega)}^{2}  \right) ds 
\end{align}


Using a Gromwall's inequality, the approximation properties \eqref{eq:ApproxProp} and the fact the  $\Vert e_{A}(0) \Vert_{L^{2}(\Omega)}^{2} \; = \; 0$ we have,


\begin{equation}
\Vert e_{A}(t) \Vert_{L^{2}(\Omega)}^{2}  + \alpha^{2}\int_{0}^{t} \Vert e_{A} \Vert_{H^{1}(\Omega)}^{2} ds  \; \leq \; C(t,\alpha^{2}, \Omega,L) \, h^{2k}
\end{equation}



where the dependency on constants dependency on $\Omega$ comes from the Trace theorem constant $C_{tr}(\Omega)$.  Then taking the maximum over all $t \in (0,T)$ we have,

\begin{equation}
\Vert e_{A} \Vert_{L^{\infty}([0,T];L^{2}(\Omega))}  \; \leq \; C(T,\alpha^{2}, \Omega,L) \, h^{k}
\end{equation}



Using the triangle inequality $\Vert u - u_{h}  \Vert_{L^{\infty}([0,T];L^{2}(\Omega))}  \; \leq \; \Vert e_{A} \Vert_{L^{\infty}([0,T];L^{2}(\Omega))} + \Vert e_{I} \Vert_{L^{\infty}([0,T];L^{2}(\Omega))}$ gives the desired results.


\newpage

\chapter{Mixed Finite Element Methods}


\noindent
The problem with the general finite element approximation of $\Phi$ is that $\nabla \Phi$ must come from post processing and we cannot control the accuracy or continuity of $\nabla \Phi$.  This is necessary not only for use in solving the drift diffusion equation, but also when we are working with a semiconductor-electrolyte system as will be explained in the next section.


\vspace{3mm}


\noindent
One way to overcome these limitations is to use a mixed finite element approach to solve Poisson's equation.  In the mixed formulation we are solving the problem as in \eqref{eq:Poisson}, but we define define the a new variable related to the electric field as:


$$ \textbf{E}= -\epsilon \, \nabla \, \Phi $$


\noindent
Then \eqref{eq:Poisson} becomes the system:


\begin{align}
\nabla \, \cdot \, \textbf{E} \, &= f \label{eq:DivEq}\\
\epsilon^{-1} \, \textbf{E} + \nabla \, \Phi &= 0 \label{eq:Efield}
\end{align}


\noindent
With the boundary conditions:


\begin{align}
\Phi \, &= \, \Phi_{D} \qquad \text{on} \quad \partial \Omega_{D} \\ 
\textbf{E} \cdot \boldsymbol \eta &= 0 \qquad \text{on} \quad \partial \Omega_{N} \label{eq:SystemNeumannBC} \\
 \partial \Omega = \partial \Omega_{D} \, \cup \, \partial \Omega_{N} \quad & \quad \partial \Omega_{D} \, \cap \, \partial \Omega_{N} = \emptyset
\end{align}


\vspace{2mm}


\noindent
Now, mixed finite element methods seek weak solutions of both $\Phi$ and $\textbf{E}$ by multiplying our \eqref{eq:DivEq} by a test function $v \in L_{2}(\Omega)$ \eqref{eq:Efield} by a test function $\textbf{p} \in H(\text{div}, \Omega)$ then integrating by parts over entire the domain for just \eqref{eq:Efield}:



  
\begin{align}
&\int_{\Omega}  \, v \, \left( \nabla  \, \cdot \, \textbf{E} \right) \, dx \, =  \, \int_{\Omega}  \,  v \, f \, dx  \\
&\int_{\Omega} \,  \textbf{p} \, \cdot \,  \epsilon^{-1} \, \textbf{E} \, dx \, 
+ \, \int_{\partial \Omega_{N}} \, \left( \textbf{p} \, \cdot \, \boldsymbol \eta \right) \, \Phi  \, dx  \,
+ \, \int_{\partial \Omega_{D}} \, \left( \textbf{p} \, \cdot \, \boldsymbol \eta \right) \, \Phi  \, dx  
-\int_{\Omega} \,  \left(\nabla \, \cdot \, \textbf{p} \right) \, \Phi \, dx = 0
\end{align}


\noindent
If we enforce the boundary conditions and the restrict $\textbf{p}
\cdot \boldsymbol \eta = 0$, on $\partial \Omega_{N}$, so that \textbf{p} and \textbf{E} are in the space function space, then we have the weak formulation:


\vspace{2mm}

\begin{center}
Find $(\Phi \, , \textbf{E} \, )\, \in \, L_{2}(\Omega) \, 
\times\, V$ such that for all $(v, \textbf{p}) \, \in \, L_{2}(\Omega) \, 
\times\, V$:
\end{center}


\begin{align}
&\int_{\Omega} \, v  \, \left( \nabla  \, \cdot \, \textbf{E} \right) \, dx \, =  \, \int_{\Omega} \, v  \, f \, dx  \\
&\int_{\Omega} \, \textbf{p} \, \cdot \, \epsilon^{-1} \, \textbf{E} \, dx \, 
+ \int_{\Omega} \, \left(\nabla \, \cdot \, \textbf{p} \right)\, \Phi \, dx 
\, = \, 
\int_{\partial \Omega_{N}} \, \left( \textbf{p} \, \cdot \, \boldsymbol \eta \right) \, \Phi_{D} \,  \, dx  \, 
\end{align}


\vspace{2mm}


\noindent
Where $V  \, = \,  \{ \, \boldsymbol \phi \, \in \, H(\text{div}\,; \, \Omega) : \, \boldsymbol \phi \, \cdot \, \boldsymbol \eta = 0 \quad \text{on} \quad \partial \Omega_{N} \} $


\vspace{2mm}


\noindent
Now we wish to project our functions onto a finite dimensional vector space.  However, we allow $\Phi$ and \textbf{E} to be in different vector spaces. Therefore we write:

\vspace{2mm}


$$ \Phi = \sum_{J} \Phi_{J} \, \psi_{J} ,  \qquad \textbf{E} = \sum_{J} E_{J} \boldsymbol \Upsilon_{J} $$


\vspace{2mm}


\noindent
Where $\psi \in L_{2}(\Omega)$ and we enforce that $\boldsymbol \Upsilon \in V$ be continuous across element nodes.  We also take $\textbf{p} = \boldsymbol \Upsilon_{I}$ and $v = \psi_{I}$ this changes the weak formulation into:


$$ \sum_{I} E_{I} \, \underbrace{\left( \int_{\Omega} (\nabla \cdot \boldsymbol \Upsilon_{I}) \,
 \psi_{J} \, dx \right)}_{A_{10}} = 
\underbrace{\int_{\Omega} \,f \, \psi_{J} \, dx}_{F}$$


\noindent
And

$$  \epsilon^{-1} \, \sum_{I} E_{i }\underbrace{\int_{\Omega} \boldsymbol \Upsilon_{I} \cdot \boldsymbol \Upsilon_{J} \, dx}_{A_{00}}
+\sum_{I} \Phi_{I} \,  \underbrace{\int_{\Omega} \, \psi _{I}
\, \left( \nabla \cdot \boldsymbol \Upsilon_{J} \right)}_{A_{01}}
=  \, \underbrace{\int_{\partial \Omega} \, \Phi_{D} \, \boldsymbol \Upsilon_{I}
\cdot \, \boldsymbol \, \boldsymbol \eta \, ds}_{V} = 0
$$

\vspace{2mm}

\noindent
Then we write the two linear systems as one larger linear system:


\vspace{2mm}


\begin{equation}
\left[ \begin{matrix}
 A_{10} & 0 \\
\epsilon^{-1} A_{00} & -A_{01} \\
\end{matrix} \right]
\left[ \begin{matrix}
\Phi_{DOF} \\
E_{DOF}
\end{matrix} \right] =
\left[ \begin{matrix}
0 \\
F  \\      
\end{matrix}\right]  + 
\left[ \begin{matrix}
V\\
0 \\
\end{matrix} \right]
\end{equation}

\vspace{2mm}

\noindent
However, this time we cannot use the \textit{Lax-Milgram theorem} to prove existence and uniqueness of the solutions.


\chapter{Local Discontinuous Galerkin Methods}
We discuss the local discontinuous Galerkin (LDG) method to approximate solutions of the Poisson equation:
\begin{equation}
\begin{aligned}
- \nabla \cdot  \left(\ \nabla u \ \right)&= f(\textbf{x}) && \mbox{in} \ \Omega, \\
-\nabla u \cdot \textbf{n}  &= g_{N}(\textbf{x}) && \mbox{on} \ \partial \Omega_{N} \\
u &= g_{D}(\textbf{x}) && \mbox{on}  \ \partial \Omega_{D}.
\end{aligned}
\end{equation}


We first introduce some notation. Let $\mathcal{T}_{h} = \mathcal{T}_{h}(\Omega) \, = \, \left\{ \, \Omega_{e} \, \right\}_{e=1}^{N}$ be  the general triangulation of a domain $\Omega \; \subset \; \mathbb{R}^{d}, \; d \, = \, 1, 2, 3$, into $N$ non-overlapping elements $\Omega_{e}$ of diameter $h_{e}$.  The maximum size of the diameters of all the elements is $h = \max( \, h_{e}\, )$.  We define $\mathcal{E}_{h}$ to be the set of all element faces and $\mathcal{E}_{h}^{i} $ to be the set of all interior faces of elements which do not intersect the total boundary $(\partial \Omega)$. We define $\mathcal{E}_{D}$ and $\mathcal{E}_{N}$ to be the sets of all element faces and on the Dirichlet and Neumann boundaries respectively. Let $\partial \Omega_{e} \in \mathcal{E}_{h}^{i}$ be a interior boundary face element, we define the unit normal vector to be,
\begin{equation}
\textbf{n} \; = \; \text{unit normal vector to } \partial \Omega_{e}  \text{ pointing from } \Omega_{e}^{-} \, \rightarrow  \, \Omega_{e}^{+}.
\end{equation}


\noindent
We take the following definition on limits of functions on element faces,
\begin{align}
w^{-} (\textbf{x} ) \, \vert_{\partial \Omega_{e} } \; = \; \lim_{s \rightarrow 0^{-}} \, w(\textbf{x}  +  s  \textbf{n}),  && w^{+} (\textbf{x} ) \, \vert_{\partial \Omega_{e} } \; = \; \lim_{s \rightarrow 0^{+}} \, w(\textbf{x}  + s  \textbf{n}).
\end{align} 
\noindent
We define the average and jump of a function across an element face as,
\begin{equation}
\{f\} \; = \; \frac{1}{2}(f^-+f^+), 
\qquad \mbox{and} \qquad 
\llbracket f \rrbracket \; = \; f^+ \textbf{n}^+ + f^- \textbf{n}^-,
\end{equation}
\noindent
and,
\begin{equation}
\{\textbf{f} \} \; = \; \frac{1}{2}(\textbf{f}^- + \textbf{f}^+), 
\qquad \mbox{and}\qquad  
\llbracket \textbf{f} \rrbracket \; = \;\textbf{f}^+ \cdot \textbf{n}^+ + \textbf{f}^- \cdot \textbf{n}^- , 
\end{equation}
\noindent
where $f$ is a scalar function and $\textbf{f}$ is vector-valued function. We note that for a faces that are on the boundary of the domain we have,
\begin{equation}
\llbracket f \rrbracket \; = \; f \,  \textbf{n} 
\qquad \mbox{and}\qquad  
\llbracket \textbf{f} \rrbracket \; = \; \textbf{f} \cdot \textbf{n}.
\end{equation}

\noindent
We denote the volume integrals and surface integrals using the $L^{2}(\Omega)$ inner products by $( \, \cdot \, , \, \cdot \, )_{\Omega}$ and $\langle  \, \cdot \, , \, \cdot \,  \rangle_{\partial \Omega}$ respectively. 
\vspace{2mm}

As with the mixed finite element method, the LDG discretization requires the Poisson equations be written as a first-order system.  We do this by introducing an auxiliary variable which we call the current flux variable $\textbf{q}$:
\begin{align}
\label{eq:Primary}
\nabla \cdot \textbf{q}_{n}
\; &= \; 
f(\textbf{x}) && \text{in} \ \Omega,  \\
\label{eq:Auxillary}
\textbf{q}_{n} 
\; &= \;
 -\nabla u && \text{in} \ \Omega,   \\
\textbf{q}_{n}   \cdot \textbf{n} 
\; &= \; g_{N}(\textbf{x}) && \text{on} \ \partial \Omega_{N},\\
u &= g_{D}(\textbf{x}) && \mbox{on}\ \partial \Omega_{D}.
\end{align}

In our numerical methods we will use approximations to scalar valued functions that reside in the finite-dimensional broken Sobolev spaces,
\begin{align}
W_{h,k}
\, &= \, 
\left\{ w \in L^{2}(\Omega) \, : \; w  \vert_{\Omega_{e}} \in \mathcal{Q}_{k,k}(\Omega_{e}), \quad \forall \, \Omega_{e}  \in \mathcal{T}_{h}  \right\}, 
\end{align}

\noindent
where $\mathcal{Q}_{k,k}(\Omega_{e})$ denotes the tensor product of discontinuous polynomials of order $k$ on the element $\Omega_{e}$. We use approximations of vector valued functions that are as,
\begin{align}
\textbf{W}_{h,k} 
\, &= \, 
\left\{  \textbf{w}  \in \left(L^{2}(\Omega)\right)^{d} \, :
 \; \textbf{w}  \vert_{\Omega_{e}} \in \left( \mathcal{Q}_{k,k}(\Omega_{e}) \right)^{d}, \quad \forall \, \Omega_{e} \in  \mathcal{T}_{h} \right\}
\end{align}

\noindent
We seek approximations for densities $u_{h} \in W_{h,k}$ and gradients $\textbf{q}_{h}\in \textbf{W}_{h,k}$. Multiplying \eqref{eq:Primary} by $w \in W_{h,k}$ and \eqref{eq:Auxillary} by $\textbf{w} \in \textbf{W}_{h,k}$ and integrating the divergence terms by parts over an element $\Omega_{e} \in \mathcal{T}_{h}$ we obtain,
\begin{align}
-
\left( \nabla w  \, , \, \textbf{q}_{h}  \right)_{\Omega_{e}}
+
\langle w \, , \,  \textbf{q}_{h}  \rangle_{\partial \Omega_{e}} 
\ &= \
\left( w , \, f \right)_{\Omega_{e}} , \nonumber \\
\left( \textbf{w} \, , \, \textbf{q}_{h} \right)_{\Omega_{e}}
-
\left(  \nabla \cdot \textbf{w} \, , \,  u_{h} \right)_{\Omega_{e}}
+
\langle   \textbf{w}  \, ,   \, u_{h}
\rangle_{\partial \Omega_{e}} 
\ &= \
0 ,\nonumber
\end{align}

\noindent
Summing over all the elements leads to the \textbf{weak formulation}:

\vspace{2mm}

\noindent
Find $u_{h} \in W_{h,k}$ and $\textbf{q}_{h} \in  \textbf{W}_{h,k} $ such that,
\begin{align}
-
\sum_{e} \left( \nabla w,  \,  \textbf{q}_{h}  \right)_{\Omega_{e}}
 +
\langle  \llbracket \,  w \, \rrbracket \, , \,  \widehat{\textbf{q}_{h} } \rangle_{\mathcal{E}_{h}^{i} }
 +
\langle  \llbracket \,  w \, \rrbracket \, , \,  \widehat{\textbf{q}_{h} } \rangle_{\mathcal{E}_{D} \cup \mathcal{E}_{N}} \ &= \  
\left( w , \, f  \right)_{\Omega} \label{eq:LDG1}  \\
\sum_{e} \left( \textbf{w} \, , \, \textbf{q}_{h} \right)_{\Omega_{e}}
-
\sum_{e} \left(  \nabla \cdot \textbf{w} , \,  u_{h} \right)_{\Omega_{e}}
+  
\langle \, \llbracket \,  \textbf{w} \, \rrbracket \, ,   \, \widehat{u_{h}}
\rangle_{\mathcal{E}_{h}^{i}}
+ 
\langle  \llbracket \,  \textbf{w} \, \rrbracket \, , \,  \widehat{u_{h}}  \rangle_{\mathcal{E}_{D} \cup \mathcal{E}_{N}} \label{eq:LDG2}
\ &= \
0
\end{align}

\noindent
for all $(w,\textbf{w}) \in W_{h,k} \times \textbf{W}_{h,k}$.


\vspace{2mm}

\noindent
The terms $\widehat{\textbf{q}_{h}}$ and $\widehat{u_{h}}$ are the numerical fluxes. The numerical fluxes are introduced to ensure consistency, stability, and enforce the boundary conditions weakly.  The flux $\widehat{u_{h}}$ is,
\begin{equation} \label{eq:UFLUX}
\widehat{u_{h}} \; = \; \left\{
\begin{array}{cl}
\left\{ u_{h} \right\} \ + \ \boldsymbol \beta \cdot \llbracket u_{h} \rrbracket &  \ \text{in} \  \mathcal{E}_{h}^{i} \\
u_{h} &  \ \text{in} \ \mathcal{E}_{N}\\
g_{D}(\textbf{x})  & \ \text{in} \ \mathcal{E}_{D} \\
\end{array}
\right.
\end{equation}


\noindent
The flux $\widehat{\textbf{q}_{h}}$ is,
\begin{equation} \label{eq:QFLUX}
\widehat{\textbf{q}_{h}}  \; = \; \left\{
\begin{array}{cl}
\left\{ \textbf{q}_{h} \right\} \ - \  \llbracket \textbf{q}_{h} \, \rrbracket \boldsymbol \beta \ + \ \sigma \, \llbracket \, u_{h} \, 
\rrbracket & \ \text{in} \ \mathcal{E}_{h}^{i} \\
g_{N}(\textbf{x}) \, \textbf{n} \,  & \ \text{in} \ \mathcal{E}_{N}\\
\textbf{q}_{h} \ + \ \sigma \, \left(u_{h} - g_{D}(\textbf{x}) \right) \, \textbf{n} & \ \text{in} \ \mathcal{E}_{D} \\
\end{array}
\right.
\end{equation}

\noindent
The term $\boldsymbol \beta$ is a constant vector which does not lie parallel to any element face in $ \mathcal{E}_{h}^{i}$.  For $\boldsymbol \beta =  0$,  $\widehat{\textbf{q}_{h}}$ and $\widehat{u_{h}}$ are called the central or Brezzi et. al. fluxes. For  $\boldsymbol \beta \neq  0$, $\widehat{\textbf{q}_{h}}$ and $\widehat{u_{h}}$ are called the LDG/alternating fluxes.  The term $\sigma$ is the penalty parameter that is defined as,
\begin{equation}
\sigma \; = \; \left\{ 
\begin{array}{cc}
\tilde{\sigma} \, \min \left( h^{-1}_{e_{1}}, h^{-1}_{e_{2}} \right) & \textbf{x} \in \langle \Omega_{e_{1}}, \Omega_{e_{2}} \rangle \\
\tilde{\sigma}  \, h^{-1}_{e} & \textbf{x} \in \partial \Omega_{e} \cap \in \mathcal{E}_{D}
\end{array}
\right. 
\label{eq:Penalty}
\end{equation}

\noindent
with $\tilde{\sigma}$ being a positive constant.

\vspace{2mm}

We can now substitute \eqref{eq:UFLUX} and \eqref{eq:QFLUX} into \eqref{eq:LDG1} and \eqref{eq:LDG2} to obtain the solution pair $(u_{h}, \textbf{q}_{h})$ to the semi-discrete LDG approximation to the drift-diffusion equation given by:

\vspace{2mm}

 
Find $u_{h} \in W_{h,k}$ and $\textbf{q}_{h} \in  \textbf{W}_{h,k}$ such that,
\begin{align}
a(\textbf{w}, \textbf{q}_{h})  \ +  \ b^{T}(\textbf{w}, u_{h}) \ &= \ G(\textbf{w}) \nonumber \\
b(w, \textbf{q}_{h}) \ + \   c(w, u_{h})  \ &= \ F(w)
\label{eq:LDG_bilinear}
\end{align}
\noindent
for all $(w, \textbf{w}) \in W_{h,k} \times \textbf{W}_{h,k}$.  This leads to the linear system,

\begin{equation}
\left[ 
\begin{matrix}
A  & -B^{T}  \\
B & C 
\end{matrix}
\right]
\left[
\begin{matrix}
\textbf{U}\\
\textbf{Q}
\end{matrix}
\right]
\ = \
\left[
\begin{matrix}
\textbf{G}\\
\textbf{F}
\end{matrix}
\right]
\end{equation}
Where $\textbf{U}$ and $\textbf{Q}$ are the degrees of freedom vectors for $u_{h}$ and $\textbf{q}_{h}$ respectively. The terms $\textbf{G}$ and $\textbf{F}$ are the corresponding vectors to $G(\textbf{w})$ and $F(w)$ respectively. The matrix in for the LDG system is non-singular for any $\sigma > 0$.

\vspace{2mm}

The bilinear forms  in \eqref{eq:LDG_bilinear} and right hand functions are defined as,

\begin{align}
b(w, \textbf{q}_{h}) \, &= \,
-
\sum_{e}  \left(\nabla w, \textbf{q}_{h} \right)_{\Omega_{e}}
+
\langle \llbracket w \rrbracket, 
\left\{\textbf{q}_{h} \right\} - \llbracket \textbf{q}_{h} \rrbracket \boldsymbol \beta \rangle_{\mathcal{E}_{h}^{i}} 
 + 
\langle w, \textbf{n} \cdot \textbf{q}_{h} \rangle_{\mathcal{E}_{D}}
\\
%%%%%%%%%%%%%%%%%%%%%%%%%%%%%%%%%%%%%%%%%%%%%%%%%%%%%%%%%%%%%%%%%%%%%%%%%%%%%%%
a(\textbf{w},\textbf{q}_{h}) \, &= \,  
\sum_{e} \left(\textbf{w}, \textbf{q}_{h} \right)_{\Omega_{e}} \\
%%%%%%%%%%%%%%%%%%%%%%%%%%%%%%%%%%%%%%%%%%%%%%%%%%%%%%%%%%%%%%%%%%%%%%%%%%%%%%%
-b^{T}(w, \textbf{q}_{h}) \, &= \,
-
\sum_{e}  \left(\nabla \cdot \textbf{w}, u_{h} \right)_{\Omega_{e}} 
+ 
\langle  \llbracket \textbf{w} \rrbracket, \left\{u_{h} \right\} + \boldsymbol \beta \cdot \llbracket u_{h} \rrbracket \rangle_{\mathcal{E}_h^{i} }
+
\langle w, u_{h} \rangle_{\mathcal{E}_{N} } \\
%%%%%%%%%%%%%%%%%%%%%%%%%%%%%%%%%%%%%%%%%%%%%%%%%%%%%%%%%%%%%%%%%%%%%%%%%%%%%%%
c(w,u_{h}) \, &= \,
\langle \llbracket w \rrbracket,  \sigma \llbracket u_{h} \rrbracket \rangle_{\mathcal{E}_{h}^{i}}
+
\langle  w, \sigma u_{h} \rangle_{\mathcal{E}_{D}}  \\
%%%%%%%%%%%%%%%%%%%%%%%%%%%%%%%%%%%%%%%%%%%%%%%%%%%%%%%%%%%%%%%%%%%%%%%%%%%%%%%
G(\textbf{w}) \ & = \ - \langle \textbf{w}, g_{D} \rangle_{\mathcal{E}_{D}}\\
%%%%%%%%%%%%%%%%%%%%%%%%%%%%%%%%%%%%%%%%%%%%%%%%%%%%%%%%%%%%%%%%%%%%%%%%%%%%%%%
F(w) \ & = \  (w,f) - \langle w, g_{N} \rangle_{\mathcal{E}_{N}}  + \langle w, \sigma g_{D} \rangle_{\mathcal{E}_{D}} 
%%%%%%%%%%%%%%%%%%%%%%%%%%%%%%%%%%%%%%%%%%%%%%%%%%%%%%%%%%%%%%%%%%%%%%%%%%%%%%%
\end{align}


\end{document}

  